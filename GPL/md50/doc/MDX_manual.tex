\documentclass{article}

\begin{document}

\title{MDynaMix Package\\
version 5.0 \\
User manual}

\author{Alexander Lyubartsev\\ \\
Division of Physical Chemistry\\
Stockholm University
}


\date{2 Jan 2007}

\maketitle


\newpage
\tableofcontents

\newpage
\section{Introduction}

MDynaMix is a general purpose molecular dynamics code for simulations
of mixtures  of rigid or flexible molecules, interacting by
AMBER-like force field in a periodic cell. The program is well suited for
simulations of flexible molecules based on the double time step algorithm.
Alternatively, rigid bonds can be treated by the SHAKE algorithm.
Algorithms for NVE, NVT and NPT ensembles are employed, as well
as Ewald summation for treatment of the electrostatic interactions. 
The program can be run both in sequential and parallel execution, in
the latter case the MPI parallel environment is required. The program is
written in standard Fortran-77 and thus can be run on any computer 
system having a Fortran compiler.
                    
The program is originally based on the MOLDYN program by Aatto Laaksonen,
available from the CCP5 program library, Daresbury Lab, UK. Since 1993 
many additional changes were made by Alexander Lyubartsev. The first
parallelized version of the program (v.4.0) was published in:

A.P.Lyubartsev, A.Laaksonen, ``{\it MDynaMix - a scalable portable parallel MD 
simulation package for arbitrary molecular mixtures}" 
Computer Physics Communications, v.128(3), pp.565-589 (2000)

Please use this reference in any work made with the help of this package.

Apart the main MDynaMix block, the package includes the following
components:

\begin{itemize}
\item
{\bf makemol} utility which provides some help in creation of
files describing molecular structure and the force field
\item
{\bf tranal} - a suite of utilities for analyzing trajectories
\item
{\bf mdee} - a version of the program which implement expanded ensemble
method for computation of free energy / chemical potential.
MDEE program is not parallelized
\end{itemize}

\subsection{Changes from previous versions}

The changes made from v.4.4 to 5.0 concern mostly the format 
of the input files: 

\begin{itemize}
\item
The main input file is written now in ``keyword - value''
manner, where the order of keywords in almost arbitrary. 

\item
The molecular description files have now extension ``.mmol''. 
Most of old .mol files can be just renamed to .mmol

\item
A few parameters of the old input file (concerning, for example, 
treatment of 1-4 neighbors) are now moved into .mmol files.
\end{itemize}

There are also some changes in functionality, see the description below.

\section{Description of the molecular structure and the force field}

\subsection{The force field}

The general form of the force field implemented in the program is:

\begin{equation}
\label{FF}
U = U_{LJ} + U_{el} + U_{bond} + U_{ang} + U_{tors} + U_{impr}
\end{equation}

where

\begin{equation}
U_{LJ} = \sum_{non-bonded}4\epsilon_{ij}\Big(\big(\frac{
\sigma_{ij}}{r_{ij}}\big)^{12}-\big(\frac{\sigma_{ij}}{r_{ij}}\big)^6\Big)
\end{equation}

\noindent
is the sum of Lennard-Jones interactions taken over non-bonded atom
pairs. A pair of atoms is considered as non-bonded if they are on
different molecules or if they are on the same molecule but separated by more
than two covalent bonds. A case when a pair of atoms is separated by exactly 
3 bonds (the so-called 1-4 neighbors) is treated separately, see below.
$r_{ij}$ is the distance between atoms $i$ and $j$.

\begin{equation}
U_{el} = \sum_{non-bonded}\frac{q_iq_j}{4\pi\epsilon_0r_{ij}}
\end{equation}

is the sum of electrostatic interactions

The remaining terms in (\ref{FF}) are inramolecular interactions
due to covalent bonds, covalent angles and torsion angles. They are 
described below.

For each molecule type used in the simulation a file describing molecular 
structure and parameters of the force field is needed. The 
file must have extension *.mmol . Some examples of .mmol files are given 
in \verb|moldb| directory.
 
.mmol files consist of several parts which are described below.
Lines, beginning with "\#" are commentaries and they are ignored by the 
program. Other lines contain parameters in free format. 

\subsection{Molecular geometry and non-bonded interaction parameters}

The first non-commentary line of a \verb|.mmol| file is the number of atoms 
in the molecule. After it the corresponding number of lines follows, one 
line per atom. Each line contains 8 compulsory
parameters. They are: 1) atom name in the program; 2),3) and 4) are the
initial X,Y,Z coordinates of the atom in the molecular coordinate system,
5) mass in atom units, 6) charge, 7) Lennard-Jones parameter $\sigma$
(effective atom diameter) in \AA, 8) Lennard-Jones parameter $\epsilon$
in kJ/M. Two optional columns may present, the 9-th column with the
chemical types of atoms according to the chosen force field and the 10th,
with the atom numbers.

This part is concluded by a reference to the molecular structure/
force field which is read by the program and printed in the output. It
consists of the first line with a number specifying the number of lines 
for the reference, followed by the corresponding number of lines with the 
reference itself.

\subsection{Bonds}

Two types of bonds are implemented, standard harmonic bond 
(Note: no factor 1/2 in the potential):

\begin{equation}
\label{harm}
U_{harm}(r) = k(r - r_0)^2
\end{equation}

and Morse potential:

\begin{equation}
\label{morse}
U_{morse}(r) = D\big(1-\exp(-\rho(r-r_0))\big)^2
\end{equation}

Harmonic bonds may be also used for description of the so-called
Urey-Bradley term which is a harmonic potential between the 1-3 neighbors
(atoms separated by two covalent bonds). This term is included for 
example into CHARMM force field as a part of interaction potential for
covalent angles.

The first line of the bond section of a \verb|.mmol| file is the number 
of bonds. Then lines, one per each bond, follow with parameters for each bond.
Each line consists of the following fields:

\begin{itemize}
\item
The first is the type of the bond potential. Available values are 0 
(harmonic bond), 1 (Morse potential) and 2 (Urey-Bradley term). In the case
of the Urey-Bradley term, the interaction is calculated exactly as in the
case of a harmonic bond (type 0), but the corresponding bond is not accounted
for definition of 1-3 and 1-4 neighbors.

\item
The second and third parameters are atom numbers which are linked by the bond. 

\item
The fourth field is the equilibrium bond length $r_0$ in \AA 

\item
The 5-th parameter is the force constant $k$
in $kJ/mol/\AA^2$  

\item
The 6-th and 7-th parameters are used only for the Morse-type bonds
and define the $D$ ($kJ/mol$) and $\rho$ ($\AA^{-1}$) parameters 
of the Morse potential respectively
\end{itemize}

\subsection{Covalent angles}

Covalent angles are described by harmonic potential of the angle:

\begin{equation}
\label{harm-ang}
U_{ang}(\theta) = k_{\theta}(\theta - \theta_0)^2
\end{equation}

\noindent
where $\theta_0$ is the equilibrium angle and $k_{\theta}$ is the force 
constant.

The first line specifies the number of angles.

Next lines give parameters of angles. The fields of each line have the 
follow meanings:

1),2) and 3) define atom numbers 

4) is the equilibrium angle in degrees

5) is the force constant in kJ/mol/rad**2

{\bf Note}. The CHARMM force field has the so-called Urey-Bradley term which is
a harmonic interaction between 1-3 neighbors. This term can be introduced 
in the list of bonds.

\subsection{Dihedral angles}. 

In this section, parameters for dihedral angles are introduced for the
standard type of torsion potential: 

\begin{equation}
\label{tors-0}
U_{tors}(\phi) = K_{\phi}(1+\cos(M\phi-\Delta))
\end{equation}

Torsional angle $\phi$ is defined as having value $180^o$ for a
trans-conformation.

Other types of dihedral angle potential are supported, they can be introduced
in other, optional, sections.

The first line of this section specifies the number of torsion angles of
this type. Lines describing torsion potential have the 
following meaning for the fields:

1)-4) define atoms for the torsion angle

5) is parameter $\Delta$ in degrees

6) is the force constant $K_{\phi}$ in $kJ/mol$

7) is the multiplicity factor $M$

Multiple torsions are supported. Any torsion angle 
can be repeated any number of times with different parameters, and all
they will be summed up in the calculations of forces and energies.

\subsection{Optional features of the force field}
 
Description of the optional features of the force field begins with 
the line defining the type of additional potential features.
These are:

\begin{itemize}

\item
\verb|tors1| 

describes torsion angles in terms of MM3 potential:

\begin{equation}
\label{tors1}
U_{tors}(\phi) = K_1(1+\cos{\phi})/2 + K_2(1-\cos{2\phi})/2 + K_3(1+\cos{3\phi})/2
\end{equation}

The first line of this section contains the number of torsions of this type
followed by lines in which fields 1)-4) define atoms and fields 5)-7) 
constants $K_1,K_2$ and $K_3$ in $kJ/M$ for each torsion angle of this type.

\item
\verb|tors5| 

describes torsion angles given by the Ryckaert-Bellemans potential:

\begin{equation}
\label{tors5}
U_{tors}(\phi) = \sum_{i=1}^5 K_i\cos^i(\phi - 180)
\end{equation}

The first line of this section contains the number of torsions of this type
followed by lines in which fields 1)-4) define atoms and fields 5)-9) 
constants $K_1,...,K_5$ in $kJ/M$ for each torsion angle of this type.

\item
\verb|improper|   

improper dihedral potential:

\begin{equation}
\label{harm-ang}
U_{impr}(\phi) = k_{\phi}(\phi - \phi_0)^2
\end{equation}

field 5) is the equilibrium angle in degrees and 6) is the force constant

\item
\verb|special| 

introduces a list, over-reading Lennard-Jones parameters for 1-4 interactions 
for the specified atoms. This list consists of the following:
 
the first line - number of atoms in the list;

other lines: the first field defines atom, the second is sigma (\AA) 
and the third parameter is epsilon in $kJ/M$.

Scaling parameter for 1-4 LJ interactions, eventually specified by 
\verb|slj14| keyword (see below), is applied to these modified LJ parameters
also.

\item
\verb|no_14|

This parameter tells that all 1-4 pairs must be excluded 
from the intramolecular Lennard-Jones and electrostatic interactions

\item
\verb|noi15|

This parameter tells that all intramolecular electrostatic and LJ
interactions are switched off

\item
\verb|sel14  <factor>|

Scale 1-4 electrostatic interactions by this factor

\item
\verb|slj14  <factor>|

Scale 1-4 Lennard-Jones interactions by this factor

\item
\verb|fSPC|

This parameter is exclusively for the flexible SPC model. It tells to the 
program to compute water intramolecular potential according to the paper 
by Toukan and Rahman,  Phys. Rev. B, 31(2) 2643 (1985)

\end{itemize}

Examples of \verb|.mmol| files are included into \verb|moldb| directory.
For big molecules, procedure of writing \verb|.mmol| files may be
somewhat automatized by the utility \verb|makemol|.

\section{The main input file}

\subsection{General notes}
Input file consists of keywords followed by parameter(s)
Lines beginning with ``\#`` are considered as commentaries.
Keywords are case sensitive.
Lines which are not recognized as keywords are considered as commentaries,
but a warning is given in the output if the first symbol of such line is not 
``\#''.  The order of keywords is almost unimportant. The only exception is
that the information on the number of molecular types 
(keyword \verb|Mol_types|) should be given before any keyword involving 
reference to specific molecules and molecular types. For missing keywords, 
default values are used.
Repeated reading of the same keyword overwrites its previous value

\subsection{Basic setup}

\begin{itemize}
\item
\verb|Main_filename  <name>|

\verb|<name>| is the base file name for the given simulation. 
Other files requested or created by the program have this name with 
various extensions
 
This keyword is compulsory.

\item
\verb|Verbose_level <number>|

\verb|<number>| - Output control parameter (integer). 
Suitable values 2-10. The less number, the less you see in the output.
Parameters higher 7 used mostly for debug purposes  

default: 5

\item
\verb|Path_DB  <name>|

\verb|<name>| - Path to the molecular database containing files describing 
molecular structures and force field parameters (.mmol files)

default:  .       (current directory)

\item
\verb|Visual  yes/no|

Run with visual shell (not included)

Default: no

\end{itemize}

\subsection{Restart control}

\begin{itemize}

\item
\verb|Read_restart  yes/no  [ASCII]|

``yes'' - read restart file \verb|<Main_filname>.dmp| and continue old 
simulation from the restart file.
.

``no''  - start a new simulation

default: yes  (this is to avoid non-intentional startup of a new simulation
which may rewrite an old restart file possibly containing results of a 
valuable simulation)

``ASCII'' is an optional parameter. If it is specified, restart file 
will be read in ASCII format 

\item
\verb|Dump_restart  <nsteps> [ASCII]|
 
Dump restart file after every \verb|<nsteps>| of the simulation

``ASCII'' is an optional parameter. If it is specified, the restart file 
will be written in ASCII format. 

Note: Binary format of restart file is preferable since it takes less 
space and do not lose any precision. ASCII restart file should be used only 
if you want to continue the simulation on a computer with another architecture

Default: do not dump restart file 

\item
\verb|Change_T  yes/no|

Change temperature after restart. This option sets up a new value for the 
target temperature which is specified in the \verb|Nose_thermostat| or 
\verb|Velocity_scaling| keywords. Otherwise (``no''), the old value of 
the temperature saved in the restart file is used

If the target temperature is changed, all the velocities are rescaled by
the corresponding factor.

Default: no

\item
\verb|Change_V  yes/no|

Change volume after restart. This option sets up a new value for the 
box sizes which are specified in the Box or Density keyword.
Otherwise (``no'') the old values of the box sizes saved in the
restart file are used

If the box sizes are changed, all the distances between the molecules 
are rescaled by the corresponding factor.

Default: no

\item
\verb|Check_only  yes/no|

``yes'' - Do not run actual simulation but only check all input 
information. If new simulation, only check input. If keyword 
\verb|Read_restart yes| is specified, type out results saved in the 
restart file.
 
``no''  - Run simulation as specified

Default: no

\item
\verb|Zero_CPU yes/no|

If ``yes'', set counter of the CPU time to zero

Default: no

\item
\verb|Zero_vel yes/no|

If ``yes'', set all velocities to zero (at start-up or at restart)

Default: no

\item
\verb|Zero_average yes/no|

If ``yes'', set all accumulators of averages to zero and clear the history
of averages

Default: no

\end{itemize}

\subsection{System}

\begin{itemize}

\item
\begin{verbatim}
Molecule_types   <num_types>
<name> <number>  [fixed]
...
<name> <number>  [fixed]
\end{verbatim}

Which molecules to simulate. \verb|<num_types>| is the number of molecule
types. This keyword must be followed by \verb|<num_types>| lines each of 
which contains at least two parameters: \verb|<name>| and \verb|<number>|

\verb|<name>| - name of the molecule. File \verb|<name>.mmol| , describing
the molecular structure and the force field parameters, must be 
present in directory specified by parameter \verb|Path_DB|.

\verb|<number>| - number of molecules of this type.

``fixed'' - Option ``fixed''  signals that 
these molecules are fixed and will not move in the MD procedure.

This keyword and its parameters (except ``fixed'') are compulsory

\item
\verb|PBC <type>|

Type of periodic boundary conditions

\verb|<type>| may be:

\verb|rect|   - rectangular (default)

\verb|hexa|   - hexagonal

\verb|octa|   - truncated octahedron

\item
\verb|Box   <box-x>  <box-y>  <box-z>|

Specifies simulation box size (in \AA). In case of hexagonal boundary 
conditions, only \verb|box-x| and \verb|box-z| have a meaning. 
For octahedral boundary conditions only \verb|<box-x>| has a meaning. 
The keyword cannot be used simultaneously with ``Density''

\item
\verb|Density  <value>|

Setup initial density. \verb|<value>| is density in $g/cm^3$. This option 
cannot be used simultaneously with setting up box sizes. In case of 
rectangular periodic boundary conditions, a cubic cell is implied. Zero 
density implies ``vacuum simulation'' regime with the box size automatically 
set to 1000 \AA 

\item
\verb|El_field <ampl> <freq>|

Put the system in a homogeneous time-dependent electrostatic field 

\begin{equation}
E(t) = A\cos(2\pi\omega t)
\end{equation}

where \verb|<ampl>|$=A$ is given in $V/cm$ and \verb|<freq> | $ = \omega$ 
in $Hz$.

\end{itemize}

\subsection{Ensemble}

\begin{itemize}

\item
\verb|Nose_thermostat   <temp>   <relax.time>|

Specify whether to use the Nose thermostat. The option is incompatible with 
``velocity scaling''. Two parameters are compulsory:

\verb|<temp>|  - target temperature

\verb|<relax.time>| - relaxation time in femtoseconds

Note: if simulation is restarted from a restart file, the old value of 
temperature specified in the restart file is normally used, ignoring value 
of this keyword. To override this, use keyword \verb|Change_T|.

\item
\verb|Velocity_scaling   <temp>   <dt>|  

Specify whether the temperature is regulated by the velocity scaling.
The option is incompatible with ``Nose thermostat''. Two
parameters are compulsory:

\verb|<temp>|  - target temperature

\verb|<dt>| - admissible deviation. If the temperature deviates from the 
target by more that ``dt'', velocities are rescaled to fit the
target temperature.
 
Note: if simulation is restarted from a restart file, the old value of 
temperature specified in the restart file is normally used, ignoring value 
of this keyword. To override this, use keyword \verb|Change_T| .

\item
\verb|Separate_thermostating  yes/no|

Specify whether to apply temperature control to the whole system (no)
or separately to each molecular type (yes). The option works both for Nose
thermostate and for temperature control by velocity scaling

default: no

{\bf Note}. At this option, the total momentum is not conserved. Use with care.
See also \verb|COM_check| keyword.

\item
\verb|COM_check  yes/no  [num]|

Specify whether to remove motion of the total center of mass.
If parameter ``num'' is zero or negative, motion of the center of
mass is removed only at the program start. If this parameter is positive, 
it gives the number of MD-steps after which the removal of center of mass
motion is repeated

Default:  yes 0 

\item
\verb|Barostate_NH <pres> <relax_time>|

Specify whether to use the Nose-Hoover barostate. Option works only if
Nose thermostat is specified. Without the keyword, constant volume
simulations are implied. Two parameters are compulsory:

\verb|<pres>| - pressure in bar (atm)

\verb|<relax_time>| - relaxation time in femtoseconds 

\item
\verb|Barostate_anisotropic  yes/no|

Specify whether to apply pressure control isotropically (no)
or separately to each direction (yes). The option has effect only
if constant-pressure simulations (barostate) is specified

default: no

\end{itemize}

\subsection{Molecular Dynamics}

\begin{itemize}
\item
\verb|Time_step  <value>|

Time step in femtoseconds. This is the long time step in case of the double 
time step algorithm. 

Default:  2 fs.

\item
\verb|Number_steps  <number>|

How many MD steps to run. 

\item
\verb|Double_timestep  <number>|

Double time step algorithm by Tuckerman et al. \verb|<number>| is the number
of short time steps in one long. The keyword cannot be used simultaneously
with \verb|Constrain|.

\item
\verb|Constrain <toler> <i_1> <i_2> <i_3> ....  <i_ntypes>|

Use SHAKE algorithm to constrain the bond lengths. The option cannot be used
with \verb|Double_timestep|. \verb|<toler>| is the tolerance level.

\verb|<i_1> <i_2> ... <i_ntypes>| are numbers 0 or 1, which specify 
whether (1) or not (0) to apply constraints to molecules of each species.

\item
\verb|R_cutoff <value>|

Cut-off (in \AA) for the Lennard-Jones and real-space part of the 
electrostatic forces.

Default:  12 \AA

\item
\verb|R_short <value>|

Cut-off (in \AA) for Lennard-Jones forces computed each short-time step
(has an effect only in double-time step algorithm)

Default:  5 \AA

\item
\verb|Neighbour_list <number>|

Update the list of neighbors (Verlet list) every \verb|<number>| steps

Default: 10

\item
\verb|Electrostatics <type>  [<A> <B>]|

How to treat electrostatics

\verb|<type>| may be Ewald (default), RF (reaction field) and Cutoff

For Ewald: A is $\alpha /R_{cutoff}$, where $\alpha$ is the Ewald convergence 
parameter. Precision of the real-space Ewald part is determined by $erfc(A)$.
$B$ defines the number of terms in the reciprocal part. It cuts the reciprocal 
series when expression in the {\it exp} of the reciprocal part exceeds $B$.
Recommended values A = 2.5 -- 3;  B = 7 -- 10.

For reaction field, A is the dielectric permittivity and B is the Debye
screening length in \AA . Setting the Debye length to 0 means an infinite 
Debye length, i.e non-conducting solution.

If \verb|<type>| is ``Cutoff'', parameters $A$ and $B$ are not necessary, and
no special treatment of electrostatic forces out of $R_{cutoff}$ takes place.

Default:  Ewald method with A=2.8 and B=9.

\item
\verb|Cut_forces   <value>|

If this option is specified and the absolute force acting on any atom 
exceeds the specified \verb|<value>|, the force will be cut to this level while
maintaining the direction. The value is given in \AA ~and has a sense
of the maximum allowed additional displacement during one time step 
( F*dt**2/m ) caused by this force

Default: Do not cut forces

\item
\verb|Combination_rule  <type>|

Use another (than Lorentz-Berthelot) combination rule for Lennard-Jones
parameters for different types

\verb|type| may be one of the following:

\verb|Sigma_geom| - LJ $\sigma$ parameter is calculated as a geometrical 
average of the both types (as for example in the OPLS force field)
 
\verb|Kong| - use Kong rules (J.Kong, J.Chem.Phys., 59, 2464, (1973))

Default: use Lorentz-Berthelot combination rules

\item
\verb|Bind_atoms   <file> <deviation>|

If this keyword is specified, atoms defined in file \verb|<file>| will be
bound to corresponding positions (given in the same file) by a
harmonic potential with characteristic deviation given by \verb|<deviation>|

\end{itemize}

\subsection{Startup}

\begin{itemize}
\item
\verb|Startup <type> [<optional parameters>]|

How to set up initial coordinates.
This option has an effect only if \verb|Read_restart| was specified as ``no'' 

\verb|<type>| may be one of the following:

\verb|xmol| (default) - the input configuration is taken from file 
\verb|<Main_filename>.start|
(see keyword ``\verb|Main_filname|'') which should be in ``XMOL'' format
The second (commentary) line of this file may contain box sizes (which 
follow after keyword \verb|BOX:|). They will be used as actual box sizes
if keyword ``\verb|Change_V|'' is not set or set to ``no''. If 
keyword ``\verb|Change_V|'' is set to ``yes'', the box sizes
specified by keywords ``\verb|Box|'' or ``\verb|Density|'' will be used. 

\verb|xyz|   - the input configuration is taken from coordinates written as 
three columns of X Y Z coordinates in \verb|<Main_filename>.start| file

\verb|Mol_COM| - molecular center-of mass coordinates are taken from 

\verb|<Main_filename>.start| file, as three columns of X Y Z coordinates. 

\verb|FCC| - molecular center of mass coordinates are put on a FCC lattice  

\verb|Cubic| - molecular center of mass coordinates are put on a cubic 
lattice  

\verb|Cyl_hole <radius>|

\verb|Sph_hole <radius>|

Molecules, except those defined in option \verb|Start_rot| are distributed
outside sphere or cylinder of radius \verb|<radius>|

In all cases, except ``xmol'' and ``xyz'', atom coordinates are build
around molecular center-of-mass using local coordinates specified in the
corresponding .mmol files.

\item
\verb|Start_rot  <i_1> <i_2> <i_3> ... <i_ntypes>|

This option has an effect only at Startup with options:

Startup option ``FCC'' or ``cubic'':

Tells which molecules rotate randomly (parameter \verb|i_n| =0 ) and 
which not  (\verb|i_n| = 1)


If Startup options ``xmol'' or ``xyz'' are specified, keyword
\verb|Start_rot| do not have any effect.

Startup options ``\verb|Cyl_hole|'' and ``\verb|Sph_hole|'':

Molecules which should be placed inside cylindrical or spherical hole,
are marked by parameter \verb|i_n| = 1. Initial coordinates of atoms of such 
molecules are taken from the corresponding .mmol file and it can be only
a single molecule of this type. Other molecules (with \verb|i_n = 0|) are 
distributed outside the spherical or cylindrical hole.

Default: all \verb|i_n|=1  (rotate molecules randomly)

\item
\verb|Gather|

Gather each molecule in one place (if its atoms were spread over different 
periodical images). Works both at start-up and at restart.


Default: use coordinates as they are.

\end{itemize}

\subsection{Properties}

\begin{itemize}

\item
\verb|Output <number>|

At output level $>=$ 5, type line(s) with some data (time step, energies, etc)
each \verb|<number>| of steps

Default: 1 

\item
\verb|Serie_average <number>|

Compute and remember intermediate averages over series of \verb|<number>| steps

Default:  10000

\item
\verb|Average_from  <number>|

Begin final averaging over series (see \verb|Serie_average| keyword) from 
series with the given \verb|<number>|

\item
\verb|Average_internal  yes/no|

Whether to compute average values of all bond lengths, angles and their
energies

Default: no


\item
\verb|Dump_XMOL  yes/no|

Dump the final configuration in ``XMOL'' format

Default: no

\item
\verb|Trajectory  <format> <step> <num_conf> <list>|

Dump trajectory. 

\verb|<format>| may be one of:

\verb|bincrd|  - binary with coordinates only

\verb|binvel|  - binary with coordinates and velocities

\verb|asccrd|  - ACSII (XMOL-format) with coordinates only    

\verb|ascvel|  - ACSII (XMOL-format) with coordinates and velocities

\verb|<step>| is time interval (in femtoseconds) between the configurations 
saved in the trajectory file

\verb|<num_conf>| - number of configurations in each trajectory file. 
After filling \verb|<num_conf>| configuration in a trajectory file, 
the program begins to write configurations in the next file. 
Trajectory files acquire extensions .001, .002, .003, etc   

\verb|<list>| - list (in the form of numbers 0 or 1 for each molecule type) 
which signals whether to include molecules of this type into trajectory.
\verb|<list>| may be set also to ``all'' (all molecules)   

\item
\verb|Bond_list yes/no|

Generate list of all bonds and put it in file bond.list

Default:no

\item
\verb|RDF_calc  <restart_option>  <RDFcutoff>  <nbins>|

Calculate RDFs.

\verb|<restart_option>| may be one of the following

RW   - read restart rdf file and dump updated rdf-restart file

RO   - only read restart RDF file but do not dump updated rdf file

(these two options have an effect only if simulation is continued from restart)

WO   - start a new collection of RDFs (that is, ignore restart file even 
if it is present) and dump a new RDF-restart file.

NOR  - do not read and do not write restart-RDF file

\verb|<RDFcutoff>| - Cutoff for RDF

\verb|<nbins>| - number of bins to compute RDFs


This keyword must be followed by a description which RDFs need to be
calculated. This part consists of the following lines (commentary symbols
are allowed):

\verb|<n>|  - number of RDFs

followed by a list of atom pairs (site numbers) for RDFs. Symbol \verb|&n|
can be used for averaging of RDFs from several atom pairs. Such a group
is counted as a single RDF.


\item
\verb|TCF_calc  <restart_option>  <N-steps>  <jump>|

Calculate TCFs.

\verb|<restart_option>| may be one of the following

RW   - read restart TCF file and dump updated tcf-restart file

RO   - only read restart TCF file but do not dump updated tcf file

(these two options have an effect only if simulation is continued from restart)

WO   - start new collection of TCFs (that is, ignore restart file even 
if it is present) and dump new restart TCF file

NOR  - do not read and do not write restart TCF file

\verb|<N-steps>| - number of steps to compute TCFs

\verb|<jump>| - number of (long) MD time steps for one step of TCF

This means that TCF will be calculated with step ``\verb|dt*jump|'' during
``\verb|dt*jump*<N-steps>|'' time

This keyword must be followed by a description which TCFs need to be
calculated. This part consists of the following lines (commentary symbols
are allowed):

A line of 12 symbols ``0'', ``1'' or ``2'' which specify which of
12 types of TCFs are to compute: 

\begin{verbatim}
  1 - velocity autocorellations
  2 - angular velocity autocorellations
  3 - 1 order Legendre polynom for dipole moment
  4 - 2 order Legendre polynom for dipole moment
  5 - 1 order Legendre polynom for reorintational tcf for a 
        specific vector
  6 - 2 order Legendre polynom for reorintational tcf for a 
        specific vector
  7 - X projection of velocity TCF
  8 - Y projection of velocity TCF
  9 - Z projection of velocity TCF
  10 - X projection of angular velocity TCF
  11 - Y projection of angular velocity TCF
  12 - Z projection of angular velocity TCF
\end{verbatim}

``0'' means do not compute, ``1'' or ``2'' signals that this type of TCF must 
be computed. "2" can be specified for tcf 7-9 or 10-12, then tcf projections 
are calculated in the molecular principal coordinate system; otherwise (1)
they are calculated in the laboratory's coordinate system

If TCF of type 5 or 6 are given for computation, then two additional 
lines should be given which specify two atom number on each molecular type
which define the unit vectors for reorientational TCF.

{\bf Note}. TCF calculations are not parallelized and therefore not 
recommended for parallel computations. Use the trajectory analysis instead.

\item
\verb|End|    

end of the input file. This line is compulsory

\end{itemize}


\section{Compilation}

The program can be installed in any user-defined directory. After
copying the distribution file into chosen directory, unpack it:

\verb|tar xfvz md50.tar.gz|

A new directory \verb|md50| arises. Change to this directory.

Before compiling, you may need to change sizes of working arrays defined
in the file \verb|dimpar.h|. The values of parameters depends both
of molecular system you want to simulate and on the computer in hands.

The are several Makefiles for different architectures. Choose the most
suitable and edit it if necessary. Usually options for optimization
can be tuned to improve performance. Then run "make" which invoke one of 
these Makefiles. The Makefiles differ mostly by compiler used and 
subroutine to count cpu time.

\subsection{Sequential execution} 

These files produce executable "md"

\verb|make default| 

calls \verb|Makefile.unknown| .  This should work for any standard f77 
compiler, with exception of accounting cpu time which is substituted by 
a ``dumb'' subroutine (substitute your own counter of cpu time in file 
\verb|cpu_dummy.f|)

\verb|make linux|

calls  \verb|Makefile.linux|. This can be  used  on Linux or other systems 
with g77 fortran  

\verb|make intel|

calls  \verb|Makefile.ifort|. This can be  used  on Linux with free (at 
least now) Intel compiler \verb|ifort|. Performance options, such as 
\verb|-xP| for processors with sse3-instructions, may substantially 
speed up the program.

\verb|make pgi|

calls \verb|Makefile.pgi|  Calls Portland Group Fortran compiler \verb|pgi|.
   

\verb|make risc|

calls \verb|Makefile.risc|   This works for IBM RISC workstations using
\verb|xlf| compiler.


\subsection{Parallel execution} 

Note that these makefiles are strongly dependent on how
the MPI library is installed. The name of executable is \verb|mdp|

\verb|make mpi|

Calls   \verb|Makefile.mpi| .  This can be used at ``standard'' MPI 
installation which uses mpif77 script to call the correct Fortran
compiler and link MPI-libraries. 

\verb|make t3e|  

Used for Cray T3E with MPI

\verb|make sp2| 

Used for  IBM SP2 parallel machine


\subsection{Other makefiles}

\indent
\verb|Makefile.DEC| - DEC alpha
  
\verb|Makefile.pgi_mpi| -  Portland Group Fortran with MPI library

\verb|Makefile.ifc| -  old Intel compiler (v.7.0 and older)
		  
\verb|Makefile.deb| - Intel fortran \verb|ifort| with debug flags. 
Produces executable \verb|mdd|

\verb|Makefile.static| - Produces static executable


\section{Execution}

The program compiled for one processor can be started from a command line:

\verb|md < md.in > md.out|

where \verb|md.in| is the main input file and \verb|md.out| is output. Names
of input and output files may be arbitrary

In the case of parallel execution, the input file must be always called 
\verb|md.input|. The program starts for example as:

\verb|mpirun -np 4 mdp > md.out|

where number of processes is 4 and \verb|md.out| is name of the output file.
Other arguments for \verb|mpirun| command can be given. Often 
\verb|mpirun| can be invoked only from a queue batch script, see
the rules in your computer center. Sometimes command \verb|poe| is used
to start a parallel program


\subsection{Files used by the program}

\begin{table}[h!]
\caption{Files used by the program. ``$<$fname$>$'' is the same for all files
in the simulation}
\begin{tabular}{|lccl|}
\hline
file   &  I/O  &  required & description \\
\hline
standard input &  I  &  yes &  the main input file\\
standard output & O  &  yes &  the main output file\\
\verb|*.mmol|   & I  &  yes &  files defining molecular structures \\
               &     &      &  and the force field \\
\verb|<fname>.start| & I & opt & start configuration \\
\verb|<fname>.dmp| & IO & opt & restart file \\
\verb|<fname>.rdf| & IO & opt & restart RDF file \\
\verb|<fname>.tcf| & IO & opt & restart TCF file \\
\verb|<fname>.xmol| & O & opt & final configuration \\
\verb|<fname>.00n| &  O & opt & trajectory files \\
\hline
\end{tabular}
\end{table}

\subsection{Memory considerations}

Parameters defining arrays boundaries are given in \verb|dimpar.h| file.
They define also the required memory.  They can be decreased to 
save memory or increased if they are not enough for the simulated system.

One of the most memory-consuming arrays is the list of the atom pairs
within the cutoff distance. Its size is defined by parameters NTOT (maximum 
number of atoms) and NBLMX - maximum number of neighbors (within cutoff) 
for each atom. Since the list of neighbors is distributed among the available 
nodes, this parameter can be decreased in the case of parallel execution. 

Another "memory consuming" parameters is MAXCF which defines the number
of point for calculation of time correlation functions.  It can be set
to 1 if calculation of time correlation functions is not carried out 
during the program run. Note also that calculation of TCF is not parallelized,
that is why it is often more appropriate for TCF calculations to dump
the trajectory and recalculate TCFs afterwords using the \verb|tranal| 
utility.

Much memory can be also taken by the array with intermediate averages for 
different parameters.  Its size is NRQS*LHIST, where NRQS is maximum number 
of calculated different averages and LHIST is the maximum number of series 
for which these averages are calculated. It may be really big 
for large macromolecules, if keyword \verb|Average_internal| is set to 
\verb|yes|. In this case, all bond lengths, covalent and torsion 
angles are calculated and remembered. 

The program checks correspondence of array boundaries to the input data and
stops if something is wrong. After correction of \verb|dimpar.h|, the code
must be recompiled.
                    

\subsection{Program structure}


The program consists of several Fortran files, each contains one
or several Fortran units. The files are:

\begin{itemize}

\item
\verb|main.f| (Part 1) - the main module.

This is the main program unit taking care of all what follows. 

\item
\verb|dimpar.h| is the file defining maximum sizes of working arrays.
The code must be recompiled after any change in this file.

\item
\verb|prcm.h|  is a header file common for most of program units.
It contains static dimensioning of the working fields and definition of 
all global parameters. Practically all modules depend somehow on it.

\item
\verb|input.f| (Part 2)  This file contains subroutines which read
the main input file and .mmol files.

\item
\verb|setup.f| (Part 3) This file setup units, reference arrays and data 
structures 

\item
\verb|mdstep.f| (Part 4) This file contains MD integration algorithms,
including thermostats and barostats as well as SHAKE algorithm for 
constraint dynamics

\item
\verb|forces.f| (Part 5) 
This file contains subroutines responsible for calculations 
of different forces. Forces are divided into two groups:
slow and fast forces. This division is essential only if
the double time step algorithm is used. Procedure for updating 
the lists of neighbors is also included into this file 

\item
\verb|mpi.f| (Part 6) contains collection of MPI calls for creating 
parallel executable code. This is the standard version of this file, 
which uses real*4 size of data during communications.

\verb|mpi_cray.f| is Cray version of MPI calls for parallel 
execution

\verb|mpi_double.f| is a version of MPI calls for parallel 
execution with double precision for the data transfer
 
\verb|mpi_safe.f| is a version of MPI calls with some additional 
controls for communications.

\verb|scalar.f| is collection of dummy MPI calls which is used in a 
single-processor version of the code

\item
\verb|restart.f| (Part 7) 

This file contains subroutines responsible for writing and
reading restart file, dumping trajectories and other operations
with input/output files

\item
\verb|aver.f| (Part 8)  collecting averages, including RDFs

\item
\verb|tcf.f| (Part 9)  calculation of time correlation functions

\item
\verb|service.f| (Part 10) is a collection of some procedures specific for 
molecular dynamics

\item
\verb|util.f| (Part 11) is a collection of some other auxiliary procedures


\item
\verb|cpu_*.f|  are different variants of counting cpu time;
\verb|getcpu.c| is a C-function to count CPU-time

		      
\end{itemize}

\section{{\em Makemol} utility}

\subsection{General organization}

The \verb|makemol| utility is used to generate a \verb|.mmol| file 
(containing information on both molecular structure and the force field)
from two files: a force field (\verb|.ff|) file, with information about 
force field parameters, and a \verb|.smol| file, which contains only 
structural information about the molecule (initial atom coordinates,
the list of covalent bonds, the partial atomic charges, and the chemical 
types of atoms (which should match the types in the force field file).
The utility constructs the list of covalent angles and torsion angles,
and pick up the force field parameters from the force field file. If some
parameters are not found in the force field file, the corresponding line
in the resulting \verb|.mmol| file is marked with an exclamation mark, 
and a note is written on the screen and in the \verb|makefile.log| file.

The list of Improper torsions is not generated automatically. If impropers
exist for the given molecule, they should be added into \verb|.mmol| file 
manually. 

The utility is run in a self-explaining dialog regime. The format of
the force field and the molecular structure files is explained below.

\subsection{The force field (.ff) file}

All the lines beginning with ``\#'' are considered as commentaries.

The file consists of a number of sections; each beginning with a keyword.
Each new keyword means the end of the old section and beginning of the 
new one corresponding to this keyword. All energies are given in $kJ/mol$
and distances in $\AA$.

\begin{itemize}

\item
\verb|BONDS|

Contains parameters for covalent bonds. Each line consists of four or six
parameters. The first two parameters are the force field atom types.
The third parameter is the force constant $k$ of the harmonic potential
(\ref{harm}) and the fourth parameter is the equilibrium
distance $r_0$. The presence of 5-th and 6-th parameters means that
the corresponding bond should be described by a Morse potential,
they have a sense of $D$ and $\rho$ parameters of the Morse potential 
(\ref{morse}) correspondingly. 

\item
\verb|ANGLES|

Contains parameters for covalent bonds. Each line consists of 5 or 7
parameters. The first three parameters are the force field atom types.
The fourth parameter is the equilibrium angle, the 5th parameter
is the force constant $k$. The 6-th and 7-th parameters are optional,
if they are present, they define the Urey-Bradley (UB) potential which
is in fact a distance-dependent harmonic potential between 1st and 3rd
atoms. The equilibrium distance for UB potential is given as a 6-th
parameter and the force constant as a 7-th. \verb|makemol| puts the 
UB term into the ``bond'' section of the \verb|.mmol| file.

\item
\verb|TORSIONS|

Contains parameters for torsion potentials of ``standard'' type (\ref{tors-0}).
Each line consists of 7 parameters, the first four define the force field
atom types, and parameters from 5-th to 7-th are $\Delta$, $K_\phi$
and $M$ correspondingly.

This section allows definition of multiple torsions and wildcards 
(denotes as ``X''). The wildcards can be used only for the 1st and 4-th
atoms in the list. The following rules are applied. If an explicit torsion   
(without wildcards) is specified, lines with wildcards matching this 
torsion are ignored. If lines with explicit torsions are repeated 
for the same set of atom types, they define multiple torsions, that 
is the total torsion potential is a sum of several terms of type 
(\ref{tors-0}).

\item
\verb|NONBONDED|

Contains Lennard-Jones parameters for non-bonded interactions.
Each line consists of 3 or 5 parameters. The first parameter is the
force field atom type. The second and third parameters are 
$\sigma$ and $\epsilon$ of the Lennard-Jones potential. If the 4-th and
5-th parameters are present, they define $\sigma$ and $\epsilon$ parameters
for 1-4 Lennard-Jones interactions.  

\end{itemize}

\subsection{The molecular structure (.smol) file}

This file contains information on molecular structure, partial atomic 
charges and the force field atom types. All the lines beginning with 
``\#'' are considered as commentaries. The first non-commentary line
defines the number of atoms in the molecule. Then follows the corresponding
number of lines describing each atom. Each line consists of 6 parameters.
The first one is the name of the atom. Parameters 2 - 4 are the initial 
X, Y and Z coordinates of atoms. The 5-th parameter is the partial atom
charge, and the 6th parameter is the force field atom type. \verb|Makemol| 
just rewrites parameters 1-5 into corresponding sections of the \verb|.mmol| 
file, and uses the force field atom type (6th parameter) in order to
find the force field parameters from the \verb|.ff| file.

\verb|Makemol| defines masses of atoms from the one or two first letters in 
the atom name (the first column of \verb|.smol| file). If the program cannot 
recognize the element, it complains about that; in such a case the mass of 
the atom must be put manually into resulting \verb|.mmol| file.

After the description of atoms, a section describing the bond structure 
follows. The first line of this section is the number of bonds. Then
the corresponding number of lines follows, with two numbers in each line 
defining numbers of atoms bound by the corresponding bond. From this list, 
\verb|Makemol| builds lists of covalent angles and torsions angles, and
substitute the corresponding force field parameters.

Besides the described above \verb|.smol| format, \verb|Makemol| can accept
\verb|alchemy| and CHARMM-coordinate formats which contain similar
information in the corresponding fields.

\section{{\em Tranal} utility}

\subsection{General organization}

TRANAL package consists of a number of utilities for analyzing of atom 
trajectories generated by molecular dynamics programs. The package
is constructed to work mainly with MDynaMix molecular dynamics program,
but is able to accept trajectories written in some other formats: 
trajectories written in XMOL format, PDB trajectories from GROMACS and 
NAMD binary trajectories.

It is supposed that in all cases, the atoms and molecules are arranged
in the following way: (the same as in MDynaMix program):

\verb|<molecules of type 1><molecules of type 2>...|

in each molecule type:

\verb|<molecule 1><molecules 2>...|

in each molecule:

\verb|<atom1><atom2>...|
(the same atom order must be in all molecules of this type)

Additionally, a term ``site'' is determined. If one chooses one molecule
of each type and set them one after another (in the same order as in the
trajectory), then the number of each atom in this sequence would correspond
to the ``site'' number. 

Each of utilities in the TRANAL suite consists of two parts: the main block 
\verb|tranal_base.f|,
which reads trajectories, and a specific part which makes required 
analysis. Correspondingly, the input file for each utility consists of 
two parts: the first one which determines how to read trajectories, and
the second part containing parameters for the required analysis.

Currently, the following analyzing utilities are included:

\begin{itemize}

\item
{\bf bileldens} - for membranes or bilayers: computation of mass density, 
electron density, and electrostatic potential across membrane

\item
{\bf diffus} - calculation of the mean square displacement and the 
self-diffusion coefficients

\item
{\bf dryrun} - just reading trajectories and optionally rewriting 
them to XMOL format

\item
{\bf latdiff} - computation of 2D mean square displacement and lateral
diffusion for membranes or bilayers

\item
{\bf order} - computation of the order parameter and angular distribution 
of specific molecular vectors 

\item
{\bf rdf} - computation of radial distribution functions

\item
{\bf sdf} - computation of spatial distribution functions

\item
{\bf torsion} - computation of distributions of torsion angles

\item
{\bf trtcf} - computation of some time correlation functions

\end{itemize}
 
Each of utilities must be compiled by a Fortran compiler together with
{\bf tranal\_base} block. For instance, the program for RDF calculations
can be compiled by:

\verb|f77 -O3 -o rdf rdf.f tranal_base.f|

Eventually, sizes of arrays may be needed to adjust, by editing {\bf tranal.h}
file, as well as sizes of arrays in each specific procedure.

Contributions for analysis of other properties are welcomed.


\subsection{tranal\_base: reading trajectories}

This part is common for all types of trajectory analysis. It reads the
first part of the input file and then reads trajectories accordingly. 
The first part of the input file is written in ``NAMELIST'' format
which looks like:

\begin{verbatim}
 $TRAJ
 parameter=value(s),
 ...
 $END
\end{verbatim}

``\verb|TRAJ|'' is the name of this NAMELIST section. 

The following parameters must be defined:

\begin{itemize}

\item
\verb|NFORM = <format>|

where \verb|<format>| is one of: 

\begin{itemize}

\item
\verb|MDYN| - MDynaMix binary trajectory (default)

\item
\verb|XMOL| - XMOL trajectory. It is implied, that the commentary 
(second) line of each configuration is written in the format:

\begin{verbatim}
(char)  <time>  (char-s) BOX:   <box_x> <box_y> <box_z>
\end{verbatim}

where \verb|(char)| is any character word, \verb|<time>| is time in $fs$,

\verb|<box_x> <box_y> <box_z>| (following after keyword \verb|BOX|) are
the box sizes. 
  
\item
\verb|PDBT| - PDB trajectory as generated by ``trajconv'' utility
of GROMACS simulation package

\item
\verb|DCDT| - DCD trajectories generated by NAMD package

\end{itemize}

\item
\verb|FNAME = <file_name>|

set the base name of the trajectory files. The trajectory must be written 
as a sequence of files \verb|<file_name>.001| , \verb|<file_name>.002| and 
so on, the largest possible number being \verb|<file_name>.999| . 

\item
\verb|PATHDB = <value>|

Directory with molecular description files (.mmol). Default is the 
current directory (.) .

\item
\verb|NTYPES = <value>|

Number of molecule types in the trajectory

\item
\verb|NAMOL = <name1> [,<name2>,...]|

\verb|NTYPES| names of molecules. It is supposed that files 
\verb|<name1>.mmol|, \verb|<name2>.mmol|,... describing the molecules are
present in the directory defined by \verb|PATHDB|. Format of .mmol files is
the same as for MDynaMix program. For analyzing trajectories generated by 
other programs, .mmol files are still needed. It is however enough to have only
the first section of .mmol files containing names of atoms.

\item
\verb|NSPEC = <n1>[,<n2>,...]|

Number of molecules of each type ( \verb|NTYPES| numbers). This parameter
is not necessary in MDynaMix binary trajectories.

\item
\verb|NFBEG = <value>|

Number of the first trajectory file (integer between 0 and 999) 

\item
\verb|NFEND = <value>|

Number of the last trajectory file (integer between 0 and 999)

\item
\verb|IPRINT = <value>|

Defines how much you see in the intermediate output. The final output with
analysis of results does not depend on it. Default value is 5.

\item
\begin{verbatim}
BOXL = <x-box-size>
BOYL = <y-box-size>
BOZL = <z-box-size>
\end{verbatim}
define the box size if it is not present in the trajectory
(implies constant-volume simulation)

\item
\verb|BREAKM = <value>|

Defines which break in trajectory (in ps) is allowed to consider the 
trajectory as continuous. The parameter is used in analysis of time-dependent 
properties.

\item
\verb|LVEL = <logical>|

Accept a logical value. \verb|.true.| signals that the trajectory 
may contain velocities, which will then be read for the analysis.

\item
\verb|LXMOL = <logical>|

Accept a logical value. If .true., rewrite trajectory in XMOL format

\verb|FXMOL = <name>|

File name for output XMOL trajectory

\item
\verb|ISTEP = <value>|

Specifies that only each \verb|ISTEP|-th configuration from the trajectory 
is taken for the analysis

\item
\verb|LBOND = <logical>|

Accept logical value. If .true., information about bonds is read 
from .mmol files

\item
\verb|VFAC = <value>| 

Multiply velocities from the trajectory by the given \verb|<value>|, 
in order to bring them to \AA/fs (the units which are used in the analysis). 
Default is 1.

\end{itemize}

\subsection{Computation of the electron density and electrostatic potential 
in bilayers: bileldens.f}

This computation is made by \verb|bileldens| utility. It is supposed
that a membrane-like system is oriented in XY-plane. The utility computes
the electron density, the mass density, the charge density, and the 
electrostatic potential across membrane in Z-direction.

Compilation:

\verb|f77 -O3 -o bileldens bileldens.f tranal_base.f|

Input parameters for this utility follow after the trajectory parameters
in the NAMELIST block \verb|ELDEN|:

\begin{verbatim}
 $ELDEN
 parameter=value(s),
 ...
 $END
\end{verbatim}

The following parameters are used:

\begin{itemize}

\item
\verb|FDENS = <filename>|

Defines name of the output file

\item
\verb|IRT = <int.num>|

Defines the type number of ``lipid'' molecules. The middle plane of the bilayer
is calculated from the center-of-mass Z-coordinate of molecules of 
this type. Eventually the molecules are transfered across the periodic 
boundaries to keep the whole bilayer within the same periodic cell.

\item
\verb|NA = <int.num>|

Defines the number of ``bins'' for computation of densities

\item
\verb|ZMAX = <value>|

Defines the borders of the interval in Z-direction. The densities and
the electrostatic potential are calculated in the range of $z$:
$-ZMAX < z < ZMAX$ relative to the membrane middle plane

\item
\verb|IATE = <list of values>|

This optional parameter defines which sites to take into account while
calculating the densities. Default is all sites. 
A value equal to 1 says that the given site is accounted while 0 says
that this site is not accounted. For example, if the 
system consists of DMPC lipids (118 atoms in molecule), water (3 atoms),
Na and Cl ions, this compute contributions from the lipids only:

\verb|IATE = 118*1,5*0|

while this compute contribution from the water oxygen only:
  
\verb|IATE = 118*0,1,0,0,0,0|

\item
\verb|LSYM = <logical>|

This parameter tells whether to symmetrize the computed densities 
relative to the membrane middle plane. Note that even slightest asymmetry
in the charge distribution may lead to a potential difference from the
both sides of membrane.

\end{itemize}

\subsection{Diffusion: diffus.f}

This procedure computes time dependence of the mean square displacement
(MSD) and evaluates the self-diffusion coefficient.

Compilation:

\verb|f77 -O3 -o diffus diffus.f tranal_base.f|

Input parameters for this utility follow after the trajectory parameters
in the NAMELIST block \verb|DIFF|:

\begin{verbatim}
 $DIFF
 parameter=value(s),
 ...
 $END
\end{verbatim}

The following parameters are used:

\begin{itemize}

\item
\verb|FILDIF = <filename>|

Defines the name of the output file

\item
\verb|IDF = <int.num>|

Defined the type of molecules for which diffusion is calculated

\item
\verb|DTT = <value>|

Defines the time interval (in $s$) for MSD calculations. It is very recommended
that this parameter is equal to the time step of the trajectory multiplied
by \verb|ISTEP| parameter defined in the trajectory (\verb|TRAJ|) section
of the input file. If the above requirement is not fulfilled, the program
may still work but less accurately. 

\item
\verb|NTT = <int.value>|

Defines the number of steps for MSD calculation. The total time of tracking 
the MSD will be thus \verb|DTT*NTT|.

\item
\verb|IAT = <int.value>|

If \verb|IAT=0|, MSD is calculated for the centers of masses of the selected
molecules. Otherwise MSD is calculated for \verb|IAT|-th atom of each 
molecule. 

\item
\verb|LCOM = <logical>|

Specifies whether to correct for the total center-of-mass motion of the
selected molecules (that is, of type \verb|IDF|). If \verb|.true.|, the
COM for each molecule is computed relative to center of mass motion of
the molecules of this type. The default is \verb|.false.|. Note also, that
correction for the center of mass motion of the whole system is not
carried out (except the case of only one molecule type and 
\verb|LCOM=.t.|). 

\item
\verb|FBEG=<value>|

Defines the beginning of linear fitting of the MSD curve to evaluate the
self-diffusion coefficient as:  \verb|FBEG*DTT*NTT| . The default value
is 0.2 , that is initial 20\% of the MSD vs time dependense is not
included. It is always recommended to look at the computed MSD vs time
dependence to evaluate acceptable value for this parameter.

\end{itemize}

{\bf Futher comments}

For each time $t$ the MSD is computed as:

\begin{equation}
\label{diffusion}
\langle \Delta r^2(t)\rangle = \langle (\vec{r}(t_0+t)-\vec{r}(t_0))^2\rangle
\end{equation}

where averaging is taken over all molecules of type \verb|IDF| and all
acceptable initial times $t_0$:   $t_{beg} \le t_0 \le t_{end}-t$, where
$t_{beg}$ and $t_{end}$ are the initial and final time of a {\it continuous}
part of the whole trajectory respectively. A trajectory is regarded as 
continuous if each next configuration differ from the previous no more
than parameter \verb|BREAKM| defined in the trajectory part (\verb|TRAJ|)
of the input file.

The output file consists of the following columns:

The first column- time $t$.

The second: for each $t$, evaluation of the diffusion coefficient as:

$$D_{av} = \frac{\langle \Delta r^2(t)\rangle}{6t}$$

The third column: for each $t$, evaluation of the diffusion coefficient as:

$$D_{dif} = \frac{1}{6}\frac{\partial\langle \Delta r^2(t)\rangle}{\partial t}$$

The 4-th column: root square of the MSD (average particle displacement)

5-th -- 7-th columns: Evaluation of diffusion coefficient in X-, Y-, and 
Z-directions as:  
  
$$D_{X} = \frac{\langle \Delta x^2(t)\rangle}{2t}$$

In all cases, diffusion is given in $10^{-5}cm^2/s$.

\subsection{Dryrun: dryrun.f}

This utility just reads the trajectories and type the coordinates of the first 
atom in the output. It can be used as a template to write new analyzing 
utilities

\subsection{Lateral diffusion: latdiff.f}

This utility computes two-dimensional mean square displacement in
membrane-like systems and the lateral diffusion coefficients. It works
essentially as \verb|diffus| utility with a few exceptions:

- parameter

\verb|IRT = <int.value>|

defines type of ``lipid'' molecules (that is molecules which build the 
membrane)

- if parameter

\verb|LCOM = .true.|

is defined, then MSD of lipids are computed relative to the center-of mass
motion of {\it each} monolayer. 

- in definition of lateral diffusion, factor 1/4 is used 
instead of 1/6 in 3D-diffusion.

\subsection{Order parameters: order.f}

This utility computes the order parameters of selected molecular vectors 
relative to the Z-axis, as well as angular distributions of these vectors.

Compilation:

\verb|f77 -O3 -o order order.f tranal_base.f|

Input parameters for this utility follow after the trajectory parameters
in the following format:

\begin{itemize}

\item
All the lines beginning with ``\#'' after the end of trajectory section
are considered as commentaries

\item
The first non-commentary line is the name of the output file

\item
The second line is an integer number.

If this parameter is zero (or negative), angular
distributions of the specified vectors are computed relative to the
positive direction of the Z axis. If this parameter is positive,
it defines the type of ``lipid'' molecules in membrane-like systems. 
The Z-coordinate of the center of mass of these molecules defines the 
membrane middle plane, and angular distributions of the molecules which 
are below the middle plane are calculated relative to the negative direction 
of the Z-axis. 

\item
The third line is the number of order parameters to compute

\item
Description of molecular vectors:

One line per each order parameter (their number is defined above)
Each line can be in one of the following form:

\verb|<n1> <n2> 0|

Here \verb|<n1>| and \verb|<n2>| are the site numbers (must belong to the 
same type of molecules) which define the molecular vector. Computation for
this vector is done independently from others

\verb|<n1> <n2> 1 <nref1> <nref2>|

followed immediately by 

\verb|<n3> <n4> |

Order parameters and angular distributions for molecular vectors 
determined by a pair of such lines
are computed together. A scalar product of vectors defined by
\verb|<n1> <n2>| and \verb|<nref1> <nref2>| is compared with scalar
product defined by \verb|<n3> <n4>| and \verb|<nref1> <nref2>|. If the
first scalar product is greater, then the vector \verb|<n1> <n2>| is
considered as going first and the vector \verb|<n3> <n4>| next; if
the second scalar product is greater, then the order in which these vectors
follow is changed to opposite (as these lines change places). Note that 
this comparison is made for each molecule separately. Such arrangement has a
sense if, for instance, two hydrogens, equivalent from the force field point
of view, are distinguishable in each specific configuration by relation to
another, ``reference'' vector, defined by numbers \verb|<nref1>,<nref2>|,
like so-called G1R and G1S order parameters in the G1 glycerol atom of
phospholipids.

\item
Resolution of the orientational distributions

The last line is an integer. If 0, no orientational distributions are 
computed. If non-zero, it defined the number of ``bins'' into which
the interval $[-1,1]$ is divided for calculation of $Cos(\theta)$
distribution.
 
\end{itemize}

{\bf further comments}.

The order parameter for a specific molecular vector is defined as

$$S = \langle \frac{3\cos^2\theta - 1}{2}\rangle$$

where $\theta$ is the angle between the molecular vector and Z-axis, and 
averaging is taken over all molecules and time frames.
The output lists also NMR dipole coupling for the corresponding vectors as:

$$C = \langle \frac{3\cos^2\theta - 1}{2|r|^3}\rangle$$
  
where $|r|$ is the length of the corresponding molecular vector.

\subsection{Radial distribution functions: rdf.f}

This utility compute radial distribution functions between specified 
atom pairs.

Compilation:

\verb|f77 -O3 -o rdf rdf.f tranal_base.f|

The input into \verb|rdf| utility consists of a NAMELIST section
\verb|RDFIN| and subsequent lines defining sites for RDF calculation.
The NAMELIST section

\begin{verbatim}
 $RDFIN
 parameter=value(s),
 ...
 $END
\end{verbatim}

contains the following parameters

\begin{itemize}

\item
\verb|FOUTRDF = <filename>|

The name of the output file

\item
\verb|NRDF = <int.value>|

number of RDFs to compute.

\item
\verb|RDFCUT = <value>|

Cut-off distance for intermolecular RDF

\item
\verb|NA = <int.value>|

Resolution of intermolecular RDF (number of bins within the 
interval \verb|[0,RDFCUT]|

\item
\begin{verbatim}
RMI = <value>
RMAX = <value>
\end{verbatim}

Boundaries for intramolecular RDFs

\item
\verb|NAI = <int.value>|

Resolution of intramolecular RDF (number of bins within the 
interval \verb|[RMI,RMAX]|

\end{itemize}

Then a number of lines follows defining the sites for RDFs. In this section 
all the lines beginning with ``\#'' are considered as commentaries. The
rules to specify RDFs are the same as in MDynaMix program.

Each (of \verb|NRDF|) RDFs is specified by a single or by several pairs of
sites. If RDF is specified by a single pair of sites, two number defining
these sites are written in the input. If an RDF needs to be specified by 
several pair of sites (in order to average them), it is written in the 
following way:

\begin{verbatim} 
&<num of pairs>
<n1-1>  <n2-1>
<n1-2>  <n2-2>
...
 
(<num of pairs> times)
\end{verbatim}

RDFs between sites \verb|<n1-1> <n2-1>| ... are averaged and counted as a
single RDF. For example, for water there exists 3 RDFs (\verb|NRDF=3|) and
the list of sites may look like (assuming O as a first atom and hydrogens
as 2 and 3):

\begin{verbatim}
1  1
&2
1  2
1  2
&3
2  2
2  3
3  3
\end{verbatim}

\subsection{Spatial distribution functions: sdf.f}

This utility computes spatial distribution functions and creates a file
which can be visualized using gOpenMol package (http://.www.csc.fi/gopenmol/).

Compilation:

\verb|f77 -O3 -o sdf sdf.f tranal_base.f|

The input into \verb|sdf| utility consists of a NAMELIST section
\verb|SDFIN| and subsequent lines defining sites which will be displayed
for visualization of the molecule around which SDF is calculated

The NAMELIST section

\begin{verbatim}
 $SDFIN
 parameter=value(s),
 ...
 $END
\end{verbatim}

contains the following parameters

\begin{itemize}

\item
\verb|FILORI = <filename>|

The name of the output file with SDF written as ``formatted'' plt-file
for gOpenMol package. Before visualization, it must be converted to the
binary using ``pltfile'' utility of gOpenMol (available also from gOpenMol 
menu).

\item
\verb|FILCRD = <filename>| 
The name of file into which average coordinates of molecule, defining the
local basis, are written. These coordinates are computed using the basis of
the molecule itself, and they in fact used for visualization of the local
basis. This file is written in ``XMOL'' format. The last column shows the
variance (average deviations) of atoms around their average positions in
the local coordinate system, which may serve also for evaluation of molecular
flexibility.

\item
\verb|NOI = <int.value>|
Defines the number of equivalent local coordinate systems on the molecule
(for example, for DNA SDFs can be defined around each of phosphate group 
used as a local basis)

\item
\begin{verbatim}
IO1 = <n1_1>,<n1_2>,...,<n1_NOI>
IO2 = <n2_1>,<n2_2>,...,<n2_NOI>
IO3 = <n3_1>,<n3_2>,...,<n3_NOI>
\end{verbatim}

\verb|NOI| values in each of these three lines. They define \verb|NOI|
local basises. Each basis is defined by three sites (for example,
\verb|<n1_1>,<n2_1>,<n3_1>|. The center of local coordinate system is set
at the first site, the X-axis goes as a mediana of 2-1-3 angle, and the
Z-axis is perpendicular to the plane defined by these three sites.

\item
\verb|NSOR = <int.value>|

Number of sites for calculation of SDF. 

\item
\verb|ISOR = <n1>,<n2> ...|

\verb|NSOR| integer values. They define SDF of which sites to compute,
and then take an average over them. 

\item
\begin{verbatim}
RXMAX = <value> 
RYMAX = <value> 
RZMAX = <value> 
\end{verbatim}
These parameters define the box inside which SDF is calculated
(for example, from \verb|-RXMAX| to \verb|RXMAX| in X-direction)

\item
\begin{verbatim}
NOMX = <value> 
NOMY = <value> 
NOMZ = <value> 
\end{verbatim}
resolution (number of bins) in each direction

\end{itemize}

After the NAMELIST section, there is a section in the input file 
which describes which atoms are displayed in the center of SDF as
a structure defining the local basis. 
The first line of this section may be either an integer number or
the word ``\verb|ALL|''. 

If an integer number is specified, it defines the number of atoms
displayed in the central structure. Then this number of lines follows,
each line per atom, each consisting of \verb|NOI + 1| values.

The first position in each line is one of numbers 1,2,3.
Positions from 2 to \verb|NOI + 1| define sites which are to be displayed,
for each of \verb|NOI| equivalent local coordinate system.

The number in the first position of each line means the following:

1 - coordinates of this atom are averaged independently of the next atom

2 - average coordinates of this atom are calculated taking into account
the next atom also (for example, if these are two hydrogens of water)

3 - average coordinates of this atom are calculated taking into account
the next two atoms (for instance, treating hydrogens in CH3 group)

Example of these lines for water:
\begin{verbatim}
1 1
2 2 
1 3
\end{verbatim}

and for methanol (see the order of atoms in CH3OH.mmol file)

\begin{verbatim}
1  1
1  2
1  3
3  4
2  5
1  6
\end{verbatim}
 
If parameter \verb|ALL| is specified in the first (and the only) line of 
this section, then all the atoms of the molecule defining the local basis 
are displayed, without any special treatment of averaging coordinates of 
equivalent atoms.

\subsection{Time correlation functions:  trtcf.f}

This utility computes some time correlation functions. For a molecular 
vector $\vec{n}(t)$, the time correlation function can be determined as:

\begin{equation}
\label{TCF}
c(t) = \frac{\langle\vec{n}(t_0)\cdot\vec{n}(t_0+t)\rangle}
{\langle\vec{n}(t_0)^2\rangle}
\end{equation}

\noindent
where averaging is taken over all molecules of the given type and all
acceptable initial times $t_0$:   $t_{beg} \le t_0 \le t_{end}-t$, where
$t_{beg}$ and $t_{end}$ are initial and final time of a {\it continuous}
part of the whole trajectory respectively. A trajectory is regarded as 
continuous if each next configuration differs from the previous by no more
than parameter \verb|BREAKM| defined in the trajectory part (\verb|TRAJ|)
of the input file.

The program can compute the following time correlation functions:

\begin{itemize}
\item
Molecular translational velocity (velocity of the center of mass) and
their X,Y,Z components in the laboratory or in the molecular coordinate systems
\item
Molecular angular velocity and
their X,Y,Z components in the laboratory or in the molecular coordinate systems
\item
Dipole moment and its second Legendre polynomial
\item
Of a chosen molecular vector and its second Legendre polynomial
\end{itemize}

The second Legendre polynomial TCF is defined as:

\begin{equation}
\label{TCF-L2}
c(t) = \frac{\langle L_2(\vec{n}(t_0)\cdot\vec{n}(t_0+t))\rangle}
{\langle L_2(\vec{n}(t_0)^2)\rangle}
\end{equation}

where $L_2(x) = 0.5(3\cos^2 x -1)$ 

\verb| |

Compilation:

\verb|f77 -O3 -o trtcf trtcf.f tranal_base.f|

Note that compilation of \verb|trtcf.f| requires, in addition to
the common header file \verb|tranal.h|, another header file \verb|trtcf.h|,
which may need to be edited in order to set array borders matching the 
studied system

Input parameters for this utility follow after the trajectory parameters
in the NAMELIST block \verb|TCF|:

\begin{verbatim}
 $TCF
 parameter=value(s),
 ...
 $END
\end{verbatim}

The following parameters are used:

\begin{itemize}

\item
\verb|FILTCF = <filename>|

Defines the name of the output file

\item
\verb|NSTEG = <int.num>|

Number of time steps in the time correlation functions

\item
\verb|DTCF = <value>|

TCF time step

It is recommended that TCF time step be equal to the trajectory time
step multiplied by parameter \verb|ISTEP| defined in the trajectory part 
of the input. The total time of tracking TCF is \verb|NSTEG*DTCF|

\item
\verb|ITCF = <i1>,<i2>,...,<i12>|

Flags (integers) specifying which TCF to compute (zero value - do not compute)
Meanings of the flags are following:

\begin{itemize}

\item
\verb|<i1>| - linear velocity of molecular center of mass

\item
\verb|<i2>| - angular molecular velocity 

\item
\verb|<i3>| - dipole moment

\item
\verb|<i4>| - second Legendre polynomial of dipole moment

\item
\verb|<i5>| - a specific molecular vector (see below)

\item
\verb|<i6>| - second Legendre polynomial of a specific vector

\item
\verb|<i7>,<i8>,<i9>| - X, Y, and Z projections of the linear molecular
velocity. The projections are calculated in the laboratory frame if 
the flags are equal to 1, and they are calculated in the principal
molecular frame determined by the axis of the inertia tensor if the
corresponding flags are equal to 2.

\item
\verb|<i10>,<i11>,<i12>| - X, Y, and Z projections of the angular molecular
velocity. The projections are calculated in the laboratory frame if 
the flags are equal to 1, and they are calculated in the principal
molecular frame determined by the axis of the inertia tensor if the
corresponding flags are equal to 2.

\end{itemize}

\item
\begin{verbatim}
N1 = <n1-1>,<n2-2>,...
N2 = <n2-1>,<n2-2>,...
\end{verbatim}

\verb|NTYPES| values in each of the two lines, one per each molecular type.
These values specify sites (two per each molecule) which define the
molecular vector whose TCF is calculated

\end{itemize}



\end{document}
